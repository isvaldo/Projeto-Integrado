\documentclass[]{article}

\usepackage[brazilian]{babel}
\usepackage[utf8]{inputenc}

\title{PI - Estruturas Discretas}
\date{}
\author{Caio Marinho Americo \\ Isvaldo \\ Josinaldo \\ Marcio \\ Carlos Renato \\ Caique}

\begin{document}

\begin{titlepage}
	\maketitle
\end{titlepage}

 
\section{Introdução}
 Nesse trabalho iremos abordar a quantidade de possibilidades que existem para a criação de nome de usuário e senha. Serão considerados as 26 letra do alfabeto, com diferenciação entre maiúscula e minúscula; 10 números, de 0 até 9; e 4 caracteres especiais: ".", "@", "\--" e "\textunderscore".
 
 
 \section{Desenvolvimento}
 \subsection{Nome de Usuário}
 
 Para a criação do nome de usuário não haverá nenhuma regra especial, só será necessário ter no mínimo 4 e no máximo 64 caracteres.
 
 \begin{math}
 \\Letras\;Mai\acute{u}sculas\;=\;26
 \\Letras\;Min\acute{u}sculas\;=\;26
 \\N\acute{u}meros\;=\;10
 \\Caracteres\;Especiais\;=\;4
 \\Total\;=\;66
 \end{math}
 
 \begin{math}
 \\4\;caracteres \rightarrow 66\cdot66\cdot66\cdot66\;=\;66^{4} 
 \\5\;caracteres \rightarrow 66\cdot66\cdot66\cdot66\cdot66\;=\;66^{5}
 \\\vdots
 \\n\;caracteres \rightarrow 66\cdot66\cdot66\cdot66\cdot \;\ldots\; \cdot66\;=\;66^{n}  
 \end{math}
 
 \begin{math}
 \\Total\;=\;\sum_{n = 4}^{64}66^{n} 
 \end{math}
 
 \subsection{Senha}
 
 Para a criação da senha será considerado necessário a presença de pelos menos uma letra maiúscula, uma minúscula e um número. Além disso será necessário ter pelo menos 6 e no máximo 64 caracteres.
 
 \begin{math}
 	\\Letras\;Mai\acute{u}sculas\;=\;26
 	\\Letras\;Min\acute{u}sculas\;=\;26
 	\\N\acute{u}meros\;=\;10
 	\\Caracteres\;Especiais\;=\;4
 	\\Total\;=\;66
 \end{math}
 
  $\\$Nesse caso, devido as considerações especiais, terão duas casas na multiplicação que será preenchidas por 26, considerando que precisa haver pelo menos uma letra maiúscula e uma minúscula, e outra que será preenchida por 10, considerando que precisa haver pelo menos um número.$\\$$\\$$\\$
 
 \begin{math}
 	\\6\;caracteres \rightarrow 26\cdot26\cdot10\cdot66\cdot66\cdot66\;=\;6760\cdot66^{3} 
 	\\7\;caracteres \rightarrow 26\cdot26\cdot10\cdot66\cdot66\cdot66\cdot66\;=\;6760\cdot66^{4}
 	\\\vdots
 	\\n\;caracteres \rightarrow 26\cdot26\cdot10\cdot66\cdot66\cdot \;\ldots\; \cdot66\;=\;6760\cdot66^{n - 3}  
 \end{math}
 
 \begin{math}
 	\\Total\;=\;6760\cdot\sum_{n = 6}^{64}66^{n - 3} 
 \end{math}

\end{document}
